\documentclass{article}
\usepackage[polish]{babel}
\usepackage[utf8]{inputenc}
\usepackage[T1]{fontenc}
\frenchspacing
\usepackage{indentfirst}
\usepackage{listings}
\author{Konrad Miziński}
\title{Projekt TKOM - Dokumentacja}
\hyphenation{XMLFile XMLElement}
\hyphenpenalty=9000

\begin{document}
\maketitle


\tableofcontents
\newpage
\section{Treść projektu}
Biblioteka java do obsługi plików konfiguracyjnych XML.

\section{Cel projektu}
Celem projektu jest dostarczenie zestawu klas języka Java pozwalających na:
\begin{itemize}
	\item Odczyt z pliku konfiguracyjnego właściwości (reprzezentowanych jako pary nazwa-wartość)
	pogrupowanych w sekcje.
	\item Dodawanie, usuwanie, modyfikacja właściwości. W szczególności tworzenie własnych obiektów 
	zawierająch zdefiniowane przez użytkownika właściwości.
	\item Serializajca i zapis do pliku zmodyfikowanych/utworzonych obiektów.
\end{itemize}

\section{Wymagania funkcjonalne}
\begin{itemize}
	\item Pliki konfiguracyjne są formatu *.xml o sztywno zdefiniowanej strukturze(opisanej 
	w dalszej części dokumentacji).
	\item Właściwości zawarte w pliku konfiguracyjnym reprezentowanie są przez pary nazwa-wartość.
	\item Właściwości są pogrupowane w sekcje.
	\item Sekcje reprezentowane są przez unikalne nazwy sekcji.
	\item Dopuszczalne są sekcje puste.
	\item Nazwy właściwości są unikalne w ramach sekcji.
	\item Wartości właściwości mogą być puste.
	\item Pliki konfiguracyjne reprezentowane są przez klasy implementujące interfejs PropertiesFile,
	sekcje przez klasy implementujące interfejs Section,
	a właściwości przez klasy imlementujące intefejs Property;
	Kody źródłowe tych interfejsów zostały zamieszczone w dalszej części dokumentacji.
	\item Dostęp do klasy reprezentującej plik właściwości można uzyskać za pomocą obiektu serwisowgo,
	udostępniającego metody odczytujące zapisujące do pliku *.xml obiekty typu PropertiesFile.
	\item Za tworzenie obiektów klasy PreprtiesFile odpowiada osobna klasa pełniąca rolę fabryki obiektów.
	\item Do biblioteki dołączona jest dokumentacja wygenerowana przez javadoc.
\end{itemize}

\section{Wymiagania niefunkcjonalne}
\begin{itemize}
	\item Biblioteka jest dostarczona w postaci archiwum typu JAR.
	\item Wraz z biblioteką jest dostarczony prosty program demonstrujący możliwości biblioteki(w postaci wykonywlnego plku JAR).
	Program bedzie pracował w trybie tekstowym.
\end{itemize}

\section{Przykładowy plik konfiguracyjny}
\lstinputlisting[language=Xml]{./example.xml}

\newpage
\section{Projekt realizacji}

\subsection{Podział na moduły}
Biblioteka  składa się z 2 podstawowych modułów:
\begin{itemize}
	\item Moduł obsługi plików xml - oddpowiada za analizę leksykalną i składniową plików xml.
	Służy do tworzenia struktur danych odpowiadających poszczególnym składowym drzewa xml.
	Odpowiada za odczyt i zapis do pliku *.xml. Jest niewidoczny dla użytkownika biblioteki.
	\item Moduł API użytkownika - dostarcza metod implementujących zadania biblioteki.
	Wykorzystuje obiekty z modułu pierwszego.
\end{itemize}

\subsection{Moduł obsługi plików xml}


\subsubsection{Analizator leksykalny(Lekser)}
Lexer ma za zadanie rozbić wczytany z pliku text na leksemy.
Na ich podstawie zostają ropoznane tokeny.
Rozróżnianie są następujące tokeny: \\
<?, ?>, <, </, >, /> ,=, ", WhiteCharacter, Letter.

\subsubsection{Analizator składniowy(Parser)}
Po analizie leksykalnej dane są poddawane analizie składniowej.
Parser stara się utworzyć drzewo składni na podstawie następującej gramatyki
(uproszczonej gramatyki drzewa xml):

\begin{verbatim} 
	Document = Prolog, [S], Element;
	Prolog = '<?', Text, S, Attributes, '?>';
	Element	= '<', Text, Attributes, [S], ('>', Content, EndTag) | '/>';
	Attributes = {S, Attribute}
	Attribute = Text, Equals, ", ExtendedText, ";
	Equals = [S], '=', [S];
	EndTag = '</', Text, [S], '>';
	Content = [S],( {Element, [S]} | ExtendedText ),[S] ; 
	Text = Letter, {Letter};
	ExtendedText = [ ExtendedLetter, {ExtendedLetter | S} ], ExtendedLetter;
	ExtendedLetter = Letter | '&amp;' | '&lt;' | '&gt;' | '&quot;' | '&apos;' ;
	S = WhiteCharacter, {WhiteCharacter};
	Letter = ? znak ascii z przedziału 0x21-0x7e bez znaków: <, >, ", ', &  ?;
	WhiteCharacter = ' ' | '\t' | '\n' | '\r';
\end{verbatim}

Elementy drzewa xml są opakowane w klasy o następujących interfejsach:
\lstinputlisting[language=Java]{../src/pl/waw/mizinski/xmlproperties/xml/XMLElement.java}
\lstinputlisting[language=Java]{../src/pl/waw/mizinski/xmlproperties/xml/XMLAttribute.java}

\subsection{Moduł API użytkownika}
Moduł ten ma za zadanie dostarczyć użytkownikowi klas pozwalających dodawać/usuwać/edytować
właściwości zawarte w pliku konfiguracyjnym.
Buduje on swoje klasy w oparciu o dane odczytne z plików i zapisane w klasach modułu pierwszego.
Podczas zapisu do pliku dane najpierw są zamienieniane na klasy plików xml(klasy modułu pierwszego),
a następnie zapisywane na dysk. Fakt korzystania z formatu xml jakie prawie całkowicie ukryty przed użytkownikiem.
Klasy wchodzące w skład tego modułu implementują następujące interfejsy:
\lstinputlisting[language=Java]{../src/pl/waw/mizinski/xmlproperties/properties/PropertiesFile.java}
\newpage
\lstinputlisting[language=Java]{../src/pl/waw/mizinski/xmlproperties/properties/Section.java}
\lstinputlisting[language=Java]{../src/pl/waw/mizinski/xmlproperties/properties/Property.java}

\end{document}