\documentclass{article}
\usepackage[polish]{babel}
\usepackage[utf8]{inputenc}
\usepackage[T1]{fontenc}
\frenchspacing
\usepackage{indentfirst}
\author{Konrad Miziński}
\title{Projekt TKOM - Dokumentacja wspępna}
\hyphenation{XMLFile XMLElement}
\hyphenpenalty=5000


\begin{document}
\maketitle

\section{Treść projektu}
Biblioteka java do obsługi plików XML.

\section{Opis funkcjonalności}
Dostarczona biblioteka powinna zawierać klasy umożliwiające dostęp do całej treści pliku xml,
tzn. do wszystkich elmentów pliku wraz z ich nazwami, wartościami i atrybutami (reprezentowanymi przez ich nazwę i wartość).

\section{Wymagania funkconalne}
\begin{itemize}
	\item Pliki xml są reprezentowane przez klasy impelementujące interfejs XMLFile,
	ale alementy dzrzewa xml przez klasy implementujące interfejs XMLElement.
	Kody źródłowe tych interfejsów zastały zamieszczone w dalszej częci dokumentacji.
	\item Dostęp do poszczególnych elementów pliku xml odbywa się na zasadzię drzewa,
	tzn dostęp do dowoleno elementu różnego od elementu nadzrzędnego,
	możliwy jest jedynie poprzez jego rodzica w drzewie xml.
	\item Na każdym poziomie drzewa xml istnieje możliwośc pobrania danego elementu w formie tekstowej
	(metoda getContent())
	\item Dostęp do klasy reprezentującej plik xml można uzyskać za pomocą obiektu typu "Factory",
	udostępniającego statyczne motody tworzące obiekty typu XMLFile.
	\item Obiekty typy XMLFile mogą być tworzone na podstawie ścieżki do pliku.xml
	lub obiektu typu java.io.File ze standardowej biblioteki Javy.
	\item W przypadku niepowodzenia utworzenia obiektu typu, rzucany jest zdefininiowany uprzednio
	wyjatek typu XMLParseException. Klasy XMLFile i XMLElement nie rzucają już żadnych wyjątków.
	\item Do biblioteki dołączona jest dokumentacja wygenerowana przez javadoc.	
\end{itemize}

\section{Wymiagania niefunkcjonalne}


\end{document}
