\documentclass{article}
\usepackage[polish]{babel}
\usepackage[utf8]{inputenc}
\usepackage[T1]{fontenc}
\frenchspacing
\usepackage{indentfirst}
\usepackage{listings}
\author{Konrad Miziński}
\title{Projekt TKOM - Dokumentacja wspępna}
\hyphenation{XMLFile XMLElement}
\hyphenpenalty=9000

\begin{document}
\maketitle

\section{Treść projektu}
Biblioteka java do obsługi plików XML.

\section{Opis funkcjonalności}
Dostarczona biblioteka powinna zawierać klasy umożliwiające dostęp do całej treści pliku xml,
tzn. do wszystkich elmentów pliku wraz z ich nazwami, wartościami i atrybutami (reprezentowanymi przez ich nazwę i wartość).

\section{Wymagania funkconalne}
\begin{itemize}
	\item Pliki xml są reprezentowane przez klasy impelementujące interfejs XMLFile,
	ale alementy dzrzewa xml przez klasy implementujące interfejs XMLElement.
	Kody źródłowe tych interfejsów zastały zamieszczone w dalszej częci dokumentacji.
	\item Dostęp do poszczególnych elementów pliku xml odbywa się na zasadzię drzewa,
	tzn dostęp do dowoleno elementu różnego od elementu nadzrzędnego,
	możliwy jest jedynie poprzez jego rodzica w drzewie xml.
	\item Na każdym poziomie drzewa xml istnieje możliwośc pobrania danego elementu w formie tekstowej
	(metoda getContent())
	\item Dostęp do klasy reprezentującej plik xml można uzyskać za pomocą obiektu typu "Factory",
	udostępniającego statyczne motody tworzące obiekty typu XMLFile.
	\item Obiekty typy XMLFile mogą być tworzone na podstawie ścieżki do pliku.xml
	lub obiektu typu java.io.File ze standardowej biblioteki Javy.
	\item W przypadku niepowodzenia utworzenia obiektu typu, rzucany jest zdefininiowany uprzednio
	wyjatek typu XMLParseException. Klasy XMLFile i XMLElement nie rzucają już żadnych wyjątków.
	\item Do biblioteki dołączona jest dokumentacja wygenerowana przez javadoc.	
\end{itemize}

\section{Wymiagania niefunkcjonalne}
\begin{itemize}
	\item Biblioteka dostarczona zostanie w postaci archiwum typu JAR.
	\item Wraz z biblioteką dostarczony zostanie prosty program demonstrujący możliwości biblioteki.
	Program bedzie pracował w trybie tekstowym.
\end{itemize}


\section{Algorytm parsowania plików xml}
Pliki xml parsowane bedą za pomocą funkcji o następujących prototypach:

\subsubsection*{Funcja parsujElement:}
\begin{enumerate}
	\item Znajdź początek najbiższego taga(''<'').
	\item Znajdć koniec najibliższego taga(''>'').
	\item Wywołaj funkcję parsujNazweIAtrybuty.
	\item Jeśli pusty element (''<.../>'') to zakończ działanie funkcji.
	\item W przeciwnym wypdaku znajdź tag kończący element (</...>)
	\item Jeśli pusty element (<...></...>) to zakończ działanie funkcji.
	\item W przeciwnym wypadku jeśli element zawiera nieotagowany tekst 
	to zapamietaj wartosć elementu i zakończ działanie funkcji.
	\item W przeciwnym wypadku wykonuj funkcję parsujElement aż do osiądniecia końcowego taga rozpatrywanego elemntu.
\end{enumerate}

\subsubsection*{Funcja parsujNazweIAtrybuty:}
\begin{enumerate}
	\item Znajdź pierwszy biały znak.
	\item Tekst przed pierwszym białym znakiem potraktuj jako nazwe elementu.
	\item Pomiń kolejne białe znaki
	\item Jeśli napotkano koniec taga(''>'' lub ''/>'') to zakończ działania funcji.
	\item W przeciwnym wypadku znajdź element ''=''
	\item Tekst od ostatnio pominiętego białego znaku do znaku równości potraktuj jako nazwe atrybutu.
	\item Pobierz znak następny po znaku "=" ('' lub ' - początek wartości atrybutu).
	\item Znajdź następny znak taki jak znak początku wartości atrybutu (odpowiedznio '' lub ')
	\item Test pomiedzy znakami '' (lub ') patraktuj jako wartość atrybutu.
	\item Powrót do kroku 3.
\end{enumerate}

\section{Postawowe intefejsy}
\lstinputlisting[language=Java]{../src/pl/waw/mizinski/xml/XMLFile.java}
\lstinputlisting[language=Java]{../src/pl/waw/mizinski/xml/XMLElement.java}
\end{document}
