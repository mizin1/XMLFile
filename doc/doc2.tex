\documentclass{article}
\usepackage[polish]{babel}
\usepackage[utf8]{inputenc}
\usepackage[T1]{fontenc}
\frenchspacing
\usepackage{indentfirst}
\usepackage{listings}
\author{Konrad Miziński}
\title{Projekt TKOM - Dokumentacja wspępna}
\hyphenation{XMLFile XMLElement}
\hyphenpenalty=9000

\begin{document}
\maketitle

\section{Treść projektu}
Biblioteka java do obsługi plików konfiguracyjnych XML.

\section{Cel projektu}
Celem projektu jest dostarczenie zestawu klas języka Java pozwalających na:
\begin{itemize}
	\item Odczyt z pliku konfiguracyjnego właściwości (reprzezentowanych jako pary nazwa-wwartość)
	pogrupowanych w sekcje.
	\item Dodawanie, usuwanie, modawyfikacja właściwości. W szczególności tworzenie własnych obiektów 
	zawierająch zdefiniowane przez użytkownika właściwości.
	\item Serializajca i zapis do pliku zmodyfikowanych/utworzonych obiektów.
\end{itemize}

\section{Wymagania funkcjonalne}
\begin{itemize}
	\item Pliki konfiguracyjne są formatu *.xml o sztywno zdefiniowanej strukturze(opisanej 
	w dalszej części dokumentacji).
	\item Właściwości zawarte w pliku konfiguracyjnym reprezentowanie są przez pary nazwa-wartość.
	\item Właściwości są pogrupowane w sekcje.
	\item Sekcje reprezentowane są przez unikalne nazwy sekcji.
	\item Dopuszczalne są sekcje puste.
	\item Nazwy właściwości są unikalne w ramach sekcji.
	\item Wartości właściwości mogą być puste.
	\item Pliki konfiguracyjne reprezentowane są przez klasy implementujące interfejs PropertiesFile,
	sekcje przez klasy implementujące interfejs Section,
	a włąściwości przez klasy imlementujące intefejs Property;
	Kody źródłowe tych interfejsów zostały zamieszczone w dalszej części dokumentacji.
	\item Dostęp do klasy reprezentującej plik właściwości można uzyskać za pomocą obiektu typu
	"Factory", udostępniającego metody tworzące obiekty typu PropertiesFile.
	Ta sama klasa odpowiada zapis obiektów PropertiesFile do pliku. 
	\item Do biblioteki dołączona jest dokumentacja wygenerowana przez javadoc.
\end{itemize}

\section{Wymiagania niefunkcjonalne}
\begin{itemize}
	\item Biblioteka dostarczona zostanie w postaci archiwum typu JAR.
	\item Wraz z biblioteką dostarczony zostanie prosty program demonstrujący możliwości biblioteki.
	Program bedzie pracował w trybie tekstowym.
\end{itemize}

\section{Przykładowy plik konfiguracyjny}
\lstinputlisting[language=Xml]{./example.xml}

\section{Analiza składni}
Składnia plików konfiguracyjnych jest uproszczoną składnią języka xml.
Piniżej zapisania w natoacji EBNF:

\begin{verbatim} 
	document ::= prolog element
	element	::=	EmptyElemTag | STag content ETag
	EmptyElemTag ::= '<'  Name (S Attribute)* S? '/>'
	Attribute ::=	 Name Eq AttValue	
	AttValue ::= '"' Text '"'
	STag ::= '<'  Name (S Attribute)* S? '>'
	ETag	::=	'</' Name S? '>'
	content	::=	(element* | Text)
	Name ::= Letter NameChar*
	Text ::= NameChar*
	Letter ::= [a-zA-Z]
	NameChar ::= Letter | [0-9] | '.' | '-' | '_' | ':' | #x20
	S ::= (#x20 | #x09 | #x0D | #x0A)+
	Eq ::= S? '=' S?
\end{verbatim}



\end{document}
