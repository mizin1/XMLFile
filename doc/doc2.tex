\documentclass{article}
\usepackage[polish]{babel}
\usepackage[utf8]{inputenc}
\usepackage[T1]{fontenc}
\frenchspacing
\usepackage{indentfirst}
\usepackage{listings}
\author{Konrad Miziński}
\title{Projekt TKOM - Dokumentacja wspępna}
\hyphenation{XMLFile XMLElement}
\hyphenpenalty=9000

\begin{document}
\maketitle


\tableofcontents
\newpage
\section{Treść projektu}
Biblioteka java do obsługi plików konfiguracyjnych XML.

\section{Cel projektu}
Celem projektu jest dostarczenie zestawu klas języka Java pozwalających na:
\begin{itemize}
	\item Odczyt z pliku konfiguracyjnego właściwości (reprzezentowanych jako pary nazwa-wartość)
	pogrupowanych w sekcje.
	\item Dodawanie, usuwanie, modyfikacja właściwości. W szczególności tworzenie własnych obiektów 
	zawierająch zdefiniowane przez użytkownika właściwości.
	\item Serializajca i zapis do pliku zmodyfikowanych/utworzonych obiektów.
\end{itemize}

\section{Wymagania funkcjonalne}
\begin{itemize}
	\item Pliki konfiguracyjne są formatu *.xml o sztywno zdefiniowanej strukturze(opisanej 
	w dalszej części dokumentacji).
	\item Właściwości zawarte w pliku konfiguracyjnym reprezentowanie są przez pary nazwa-wartość.
	\item Właściwości są pogrupowane w sekcje.
	\item Sekcje reprezentowane są przez unikalne nazwy sekcji.
	\item Dopuszczalne są sekcje puste.
	\item Nazwy właściwości są unikalne w ramach sekcji.
	\item Wartości właściwości mogą być puste.
	\item Pliki konfiguracyjne reprezentowane są przez klasy implementujące interfejs PropertiesFile,
	sekcje przez klasy implementujące interfejs Section,
	a właściwości przez klasy imlementujące intefejs Property;
	Kody źródłowe tych interfejsów zostały zamieszczone w dalszej części dokumentacji.
	\item Dostęp do klasy reprezentującej plik właściwości można uzyskać za pomocą obiektu typu
	"Factory", udostępniającego metody tworzące obiekty typu PropertiesFile.
	Ta sama klasa odpowiada za zapis obiektów PropertiesFile do pliku. 
	\item Do biblioteki dołączona jest dokumentacja wygenerowana przez javadoc.
\end{itemize}

\section{Wymiagania niefunkcjonalne}
\begin{itemize}
	\item Biblioteka dostarczona zostanie w postaci archiwum typu JAR.
	\item Wraz z biblioteką dostarczony zostanie prosty program demonstrujący możliwości biblioteki.
	Program bedzie pracował w trybie tekstowym.
\end{itemize}

\section{Przykładowy plik konfiguracyjny}
\lstinputlisting[language=Xml]{./example.xml}

\newpage
\section{Projekt realizacji}

\subsection{Podział na moduły}
Biblioteka bedzie składała się z 2 podstawowych modułów:
\begin{itemize}
	\item Moduł obsługi plików xml - oddpowiada za analizę leksykalną i składniową plików xml.
	Służy do tworzenia struktur danych odpowiadających poszczególnym składowym drzewa xml.
	Odpowiada za odczyt i zapis do pliku *.xml. Jest niewidoczny dla użytkownika biblioteki.
	\item Moduł API użytkownika - dostarcza metod implementujących zadania biblioteki.
	Wykorzystuje obiekty z modułu pierwszego.
\end{itemize}

\subsection{Moduł obsługi plików xml}


\subsubsection{Analizator leksykalny(Lekser)}
Lexer ma za zadanie rozbić wczytany z pliku text na leksemy.
Na ich podstawie zostają ropoznane tokeny, którym w szczególnych przypadkach przypisywane są wartości.
W poniższej tabeli zamieszczono rozpoznawane przez lexer tokeny wraz z przykładami dla podanego fragmentu pliku xml: 
\begin{verbatim}
<?xml version="1.0" encoding="ASCII"?>
...
<property name="nazwa">value1</property>
<property/>
\end{verbatim} 

\begin{center}
    \begin{tabular}{ | l | l | l |}
    \hline  
    Leksem & Token & Wartość \\ \hline 
    <property & OpenStartTag & property\\ \hline 
    <$\backslash$property & OpenEndTag & property\\ \hline 
    > & CloseTag &\\ \hline 
    $\backslash$> & CloseEmptyElementTag & \\ \hline 
    name & AttributeName & name \\ \hline
    = & Equals & \\ \hline
    ''nazwa'' & AttributeValue & nazwa \\ \hline
    value1 & Text & value1  \\ \hline
    <?xml version=''1.0'' encoding=''ASCI''?> & Prolog &  \\ \hline
    \end{tabular}
\end{center}

Nazwy elemetów i atrybutów powinny spełniać następujące wyrażenie\\
regularne:
\begin{verbatim}
[a-ZA-Z]([a-zA-Z] | [0-9] | '.' | '-' | '_')*
\end{verbatim} 
Zaś wartości elementów(token ''Text''):
\begin{verbatim}
([a-zA-Z] | [0-9] | '.' | '-' | '_')*
\end{verbatim}
Przyporządkowanie lexemów do tokenów nie zależy tylko od spełnienia wyrażenia regularnego, ale również od stanu w jakim znajdował się lexer po dopasowaniu ostatniego lexema
(np. rozpoznanie tokena ''Text'' może mastąpić tylko jeśli poprzednio został rozpoznany token ''CloseTag'').

\subsubsection{Analizator składniowy(Parser)}
Po analizie leksykalnej dane są poddawane analizie składniowej.
Parser stara się utworzyć drzewo składni na podstawie następującej gramatyki
(uproszczonej gramatyki drzewa xml):

\begin{verbatim} 
	Document = Prolog, Element;
	Element	= EmptyElementTag | StartTag, Content, EndTag;
	EmptyElementTag = OpenStartTag, {Attribute}, CloseEmptyElementTag;
	Attribute = AttributeName, Equals, AttributeValue;
	StartTag = OpenStartTag, {Attribute}, CloseTag;
	EndTag = OpenEndTag, CloseTag;
	Content	= {Element} | Text;
\end{verbatim}
	
Elementy drzewa xml będą opakowane w klasy o następujących interfejsach:
\lstinputlisting[language=Java]{../src/pl/waw/mizinski/xmlproperties/xml/XMLElement.java}
\lstinputlisting[language=Java]{../src/pl/waw/mizinski/xmlproperties/xml/XMLAttribute.java}

\subsection{Moduł API użytkownika}
Moduł ten ma za zadanie dostarczyć użytkownikowi klas pozwalających dodawać/usuwać/edytować
właściwości zawarte w pliku konfiguracyjnym.
Wykonuje on swoje zadania poprzez delegację do obiektów modułu pierwszego,
ukrywając tym samym przez użytkownikiem strukturę pliku xml.
Klasy wchodzące w skład tego modułu bedą implementowały następujące interfejsy:
\lstinputlisting[language=Java]{../src/pl/waw/mizinski/xmlproperties/properties/PropertiesFile.java}
\newpage
\lstinputlisting[language=Java]{../src/pl/waw/mizinski/xmlproperties/properties/Section.java}
\lstinputlisting[language=Java]{../src/pl/waw/mizinski/xmlproperties/properties/Property.java}

\end{document}